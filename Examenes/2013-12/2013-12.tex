\documentclass{article}


\usepackage{srcltx}
\usepackage {graphicx}

\usepackage {amssymb,amsfonts,amsthm,srcltx}
\usepackage {amsmath}
\usepackage {longtable}

\usepackage[spanish]{babel}

\setlength{\parindent}{0pt}

\textheight 26cm \textwidth 17cm \topmargin -10mm \oddsidemargin
-3mm \evensidemargin -3mm
\renewcommand{\baselinestretch}{1.2}
\newcommand{\inc}{\subseteq_{\!\!\!\! / }}
\newcommand{\inte}{\mbox{$^{\!\circ}$}}
\newcommand{\N}{\mbox{$I\!\!N$}}
\newcommand{\Z}{\mbox{$Z\!\!\!Z$}}
\newcommand{\Q}{\mbox{$I\:\!\!\!\!\!Q$}}
\newcommand{\R}{\mbox{$I\!\!R$}}
\newcommand{\K}{\mbox{$I\!\!K$}}
\newcommand{\comp}{{\scriptstyle \circ}}
\newcommand{\indic}{\mbox{$1\!\!{\rm I}$}}
\newcommand{\sgn}{{\rm sgn\,}}
\newcommand{\be}{\begin{enumerate}}
\newcommand{\ee}{\end{enumerate}}
\newcommand{\esp}{\vspace*{5mm}}
\newcommand{\pirulo}[1]{\mbox{$#1^{^{^{\!\!\!\!\!\!\!\!\!\longrightarrow}}}$}}
\newcommand{\Po}[1]{\mbox{${\cal P}_{#1}$}}
\newcommand{\M}[2]{\mbox{${\cal M}(\R)_{#1 \times #2}$}}
\newcommand{\MC}[2]{\mbox{${\cal M}(\C)_{#1 \times #2}$}}
\newcommand{\MK}[2]{\mbox{${\cal M}(\K)_{#1 \times #2}$}}
\newcommand{\C}{\mbox{$I\:\!\!\!\!\!C$}}
\newcommand{\cant}{\mbox{$= \!\!\!\!\!\! / \!\! / \,$}}
\newcommand{\noin}{\mbox{$\,\, \in\!\!\!\!\!/ \,\,$}}
\newcommand{\base}{\mbox{$\stackrel{b}{\longrightarrow}$}}
\newcommand{\gene}{\mbox{$\stackrel{g}{\longrightarrow}$}}
\newcommand{\vetur}[2]{\mbox{$\left( \begin{array}{c} #1 \\ #2 \end{array} \right)$}}
\newcommand{\vetor}[3]{\mbox{$\left( \begin{array}{c} #1 \\ #2 \\ #3 \end{array} \right)$}}
\newcounter{cuenta}
\newcommand{\opcion}{\stepcounter{cuenta}{\bf {\rm (\thecuenta)}}\mbox{$\;\;\;$}}
\newcommand{\cuentafin}{\setcounter{cuenta}{0}}
\newtheorem{teo}{Teorema}
\newtheorem{cor}[teo]{Corolario}
%\newtheorem{criterion}[theorem]{Criterion}

%\newtheorem{exercise}[theorem]{Exercise}

\newtheorem{Ejercicio}{Ejercicio}
\newenvironment{ejercicio}{\begin{Ejercicio} \rm\small}{\rm\end{Ejercicio}}

\newtheorem{lema}[teo]{Lema}
\newtheorem{nota}[teo]{Notaci{\'o}n}
%\newtheorem{problem}[theorem]{Problem}
\newtheorem{prop}[teo]{Proposici{\'o}n}

%\newtheorem{remarks}[theorem]{Observaciones}
%\newtheorem{solution}[theorem]{Solution}
%\newtheorem{summary}[theorem]{Summary}
\theoremstyle{definition}
\newtheorem{df}[teo]{Definici{\'o}n}
\newtheorem{ejemplo}[teo]{Ejemplo}
\newtheorem{ejemplos}[teo]{Ejemplos}
\newtheorem{obs}[teo]{Observaci{\'o}n}
%\input tcilatex

\begin{document}

\begin{flushleft} 
Universidad de la Rep\'ublica\\
Facultad de Ingenier\'\i a.
\end{flushleft}

\begin{center}
Examen de Matem\'atica Discreta II\\
.... de diciembre de 2013\\

\vspace{0.1cm}

\begin{tabular}{|c|c|c|}
\hline
N\'umero de Examen &\hspace{1.0cm} C\'edula \hspace{1.2cm} & \hspace{1.3cm} Nombre y Apellido \hspace{1.5cm} \\
\hline
 & & \\
 & & \\
\hline
\end{tabular}
\end{center}

\vspace{0.3cm}


\begin{enumerate}

\vspace{2mm}

\item\textbf{($aa$ puntos)}
\begin{enumerate}
\item Hallar todas las soluciones posibles con $a,b \in \mathbb{N}$ de
  \begin{itemize} 
    \item $a+b = 1235$
    \item mcm$(a,b) = 714$ mcd$(a,b).$
%\textbf{Soluci\'on: $(d = 19, a = 969, b = 266)$}.
\end{itemize}
  \item ?`Qu\'e restos puede dejar un cubo perfecto al dividir entre $(d-10)$? (siendo $d=$ mcd$(a,b)$ de la parte anterior).
  \item Mostrar que la ecuaci\'on $x^3-117y^3=5$ no tiene soluciones enteras.
%\newline\emph{Sugerencia: Si la ecuaci\'on tuviese soluciones enteras mirar la ecuaci\'on m\'odulo $9$ y llegar a un absurdo.}
\end{enumerate}

\vspace{2mm}

\item
\textbf{($bb$ puntos)}
\begin{enumerate}
  \item Sea $f: (G_1, *) \longrightarrow (G_2, \star)$ un morfismo de grupos. Definir Ker$(f)$ y demostrar que Ker$(f)$ es un subgrupo normal de $G_1$.
  
  \item  Sea $R(x$ el grupo de las funciones racionales con el producto.............

  \item Sea $f: (\Z, +) \longrightarrow (\Z, +)$ un morfismo de grupos, 
  \begin{enumerate}
    \item[i.] Demostrar que $h: \Z \longrightarrow R(x)$ , tal que $h(n) = x^{f(n)}$ es un morfismo de grupos.
    \item[ii.] Hallar el Ker$(h)$.
  \end{enumerate}
  \item  Sabiendo que $h(-1) = \frac{1}{x^{a}}$ donde $a$ es la menor ra\'iz primitiva de $U(17)$, describir el morfismo $f$.
\end{enumerate}

\vspace{0.2cm}

\item
\textbf{($cc$ puntos)}
\begin{enumerate}
  \item Mostrar que $3$ es ra\'iz primitiva m\'odulo $31$.
  \item Calcular $\displaystyle \sum_{i=0}^{309} 3^{i} \mbox{ m\'od} (31)$
\end{enumerate}

\item
\textbf{($dd$ puntos)}
\begin{enumerate}
\item Describir el Criptosistema RSA.
\item Definir la funci\'on dde desencriptado y demostrar que desencripta.

\end{enumerate}

\end{enumerate}

\end{document}
